% Don't touch this %%%%%%%%%%%%%%%%%%%%%%%%%%%%%%%%%%%%%%%%%%%
\documentclass[11pt]{article}
\usepackage{fullpage}
\usepackage[left=1in,top=1in,right=1in,bottom=1in,headheight=3ex,headsep=3ex]{geometry}
\usepackage{graphicx}
\usepackage{float}
\usepackage{makecell}

\newcommand{\blankline}{\quad\pagebreak[2]}
%%%%%%%%%%%%%%%%%%%%%%%%%%%%%%%%%%%%%%%%%%%%%%%%%%%%%%%%%%%%%%

% Modify Course title, instructor name, semester here %%%%%%%%

\title{BIOS 635: Introduction to Machine Learning}
\author{Kevin Donovan}
\date{Spring 2021}

%%%%%%%%%%%%%%%%%%%%%%%%%%%%%%%%%%%%%%%%%%%%%%%%%%%%%%%%%%%%%%

% Don't touch this %%%%%%%%%%%%%%%%%%%%%%%%%%%%%%%%%%%%%%%%%%%
\usepackage[sc]{mathpazo}
\linespread{1.05} % Palatino needs more leading (space between lines)
\usepackage[T1]{fontenc}
\usepackage[mmddyyyy]{datetime}% http://ctan.org/pkg/datetime
\usepackage{advdate}% http://ctan.org/pkg/advdate
\newdateformat{syldate}{\twodigit{\THEMONTH}/\twodigit{\THEDAY}}
\newsavebox{\MONDAY}\savebox{\MONDAY}{Mon}% Mon
\newcommand{\week}[1]{%
%  \cleardate{mydate}% Clear date
% \newdate{mydate}{\the\day}{\the\month}{\the\year}% Store date
  \paragraph*{\kern-2ex\quad #1, \syldate{\today} - \AdvanceDate[4]\syldate{\today}:}% Set heading  \quad #1
%  \setbox1=\hbox{\shortdayofweekname{\getdateday{mydate}}{\getdatemonth{mydate}}{\getdateyear{mydate}}}%
  \ifdim\wd1=\wd\MONDAY
    \AdvanceDate[7]
  \else
    \AdvanceDate[7]
  \fi%
}
\usepackage{setspace}
\usepackage{multicol}
%\usepackage{indentfirst}
\usepackage{fancyhdr,lastpage}
\usepackage{url}
\pagestyle{fancy}
\usepackage{hyperref}
\usepackage{lastpage}
\usepackage{amsmath}
\usepackage{layout}

\lhead{}
\chead{}
%%%%%%%%%%%%%%%%%%%%%%%%%%%%%%%%%%%%%%%%%%%%%%%%%%%%%%%%%%%%%%

% Modify header here %%%%%%%%%%%%%%%%%%%%%%%%%%%%%%%%%%%%%%%%%
\rhead{\footnotesize BIOS 635 Syllabus}

%%%%%%%%%%%%%%%%%%%%%%%%%%%%%%%%%%%%%%%%%%%%%%%%%%%%%%%%%%%%%%
% Don't touch this %%%%%%%%%%%%%%%%%%%%%%%%%%%%%%%%%%%%%%%%%%%
\lfoot{}
\cfoot{\small \thepage/\pageref*{LastPage}}
\rfoot{}

\usepackage{array, xcolor}
\usepackage{color,hyperref}
\definecolor{clemsonorange}{HTML}{EA6A20}
\hypersetup{colorlinks,breaklinks,linkcolor=clemsonorange,urlcolor=clemsonorange,anchorcolor=clemsonorange,citecolor=black}

\begin{document}

\maketitle

\blankline

\begin{tabular*}{.93\textwidth}{@{\extracolsep{\fill}}lr}

%%%%%%%%%%%%%%%%%%%%%%%%%%%%%%%%%%%%%%%%%%%%%%%%%%%%%%%%%%%%%%

% Modify information %%%%%%%%%%%%%%%%%%%%%%%%%%%%%%%%%%%%%%%%%
E-mail: \texttt{kmdono02@ad.unc.edu} & Web: \href{https://kmdono02.github.io/}{\tt\bf https://kmdono02.github.io/}  \\

 Office Hours: W 10-11am  &  Class Hours: T/Th 9:30-10:45am \\

 Office/Classroom: \href{https://uncsph.zoom.us/j/8587296876}{\tt\bf https://uncsph.zoom.us/j/8587296876} \\
\hline
\end{tabular*}

\vspace{5 mm}

\section*{Course Description}
This course is an introductory course to machine learning and statistical learning and is required for MPH students with Data Science concentration. While some technical details will be covered, emphasis will be made on understanding the models, intuitions, and strengths and weaknesses of the various approaches.  The primary goal is to provide conceptual understanding of machine learning methods and how these relate back to statistical principles.  The secondary goal is to equip students with knowledge of existing tools for data analysis and to get students prepared for more advanced courses in machine learning. Good research principles with respect to machine learning analyses will also be covered (reproducibility, legible and complete code, etc.).  Programming language will be R – students will learn how to use the free and powerful software R in connection with each of the methods exposed in the class. For deep learning, Keras/TensorFlow in Python will be introduced if time permits.

\section*{Prerequisites}
Prerequisites: BIOS 611, BIOS 550, R programming language, college-statistics course

\section*{Instructor}
Kevin Donovan\\
PhD Candidate\\
Department of Biostatistics\\
Phone: 315-727-3603\\
Email: kmdono02@ad.unc.edu

\section*{Teaching Assistant}
Emily Damone\\ 
Email: edamone@live.unc.edu\\
Office Hours: Monday 3:30-4:30PM

\section*{Course Materials}
\subsection*{Course Website}
GitHub: \href{https://github.com/kmdono02/BIOS_635}{\tt\bf https://github.com/kmdono02/BIOS\_635}\\
Slack: \href{...}{\tt\bf ...}
\subsection*{Course Texts}
Required Textbook:
\begin{itemize}
\item James, Witten, Hastie, and Tibshirani, An Introduction to Statistical Learning, Springer (\href{https://statlearning.com/}{\tt\bf FREE}).
\end{itemize}
Recommended Textbooks:
\begin{itemize}
\item Grolemund and Wickham, R for Data Science, O’Reilly (\href{https://r4ds.had.co.nz/}{\tt\bf FREE}).
\item Irizarry, Introduction to Data Science: Data Analysis and Prediction Algorithms in R, CRC Press (\href{https://rafalab.github.io/dsbook/}{\tt\bf FREE}).
\end{itemize}
Advanced Textbooks:
\begin{itemize}
\item Hastie, Tibshirani, Friedman, The Elements of Statistical Learning, Springer (\href{https://web.stanford.edu/~hastie/ElemStatLearn/}{\tt\bf FREE}).
\item Efron and Hastie, Computer Age Statistical Inference, Cambridge University Press (\href{https://web.stanford.edu/~hastie/CASI_files/PDF/casi.pdf}{\tt\bf FREE}).
\end{itemize}

\section*{Course Format}
The course format will include two weekly lectures via video call interfacing through Zoom at the provided link on page 1. The lecture will be supplemented with in-class exercises, case studies and examples for real data set analysis.  Published research in academic journals and non-academic sources will be discussed and critical evaluated.

\section*{Course Objectives}
\begin{enumerate}
\item Gain conceptual understanding of machine learning through a statistical framework
\item Be introduced to standard machine learning methods and when each is most appropriate to use in the analysis
\item Learn how to implement these methods using R (and Python if time permits) through real data analysis
\item Understand complications with applying machine learning methods to data and in society at large (e.g. generalizability, data ownership, group representation in data, etc.)
\item Gain skills in critically evaluating published academic and non-academic research which use machine learning methods
\end{enumerate}

\section*{Course Policies and Resources}
\subsection*{Recognizing, Valuing and Encouraging Inclusion and Diversity in the Classroom}
We share the School`s \href{https://sph.unc.edu/resource-pages/inclusive-excellence/diversity}{\tt\bf commitment to diversity}.  We are committed to ensuring that the School is a diverse, inclusive, civil and welcoming community. Diversity and inclusion are central to our mission — to improve public health, promote individual well-being and eliminate health inequities across North Carolina and around the world.  Diversity and inclusion are assets that contribute to our strength, excellence and individual and institutional success. We welcome, value 
and learn from individual differences and perspectives. These include but are not limited to: cultural and racial/ethnic background; country of origin; gender; age; socioeconomic status; physical and learning abilities; physical appearance; religion; political perspective; sexual identity and veteran status. Diversity, inclusiveness and civility are core values we hold, as well as characteristics of the 
School that we intend to strengthen. 

We are committed to expanding diversity and inclusiveness across the School—among faculty, staff, students, on advisory groups, and in our curricula, leadership, policies and practices. We measure diversity and inclusion not only in numbers, but also by the extent to which students, alumni, faculty and staff members perceive the School’s environment as welcoming, valuing all individuals and supporting their development.” In this class, we practice these commitments in the following ways: 
\begin{itemize}
\item Develop classroom participation approaches that acknowledge the 
diversity of ways of contributing in the classroom and foster 
participation and engagement of all students.
\item Structure assessment approaches that acknowledge different methods for acquiring knowledge and demonstrating proficiency.
\item Encourage and solicit feedback from students to continually improve inclusive practices. 
\end{itemize}
As a student in the class, you are also expected to understand and uphold the following UNC policies:
\begin{itemize}
\item \href{http://sph.unc.edu/resource-pages/diversity/}{\tt\bf Diversity and Inclusion at the Gillings School of Global Public Health}
\item \href{http://policy.sites.unc.edu/files/2013/04/nondiscrim.pdf}{\tt\bf UNC Non-Discrimination Policies}
\item \href{https://deanofstudents.unc.edu/incident-reporting/prohibited-harassmentsexual-misconduct}{\tt\bf Prohibited Discrimination, Harassment, and Related Misconduct at UNC}
\end{itemize}   

\subsection*{Accessibility}
UNC-CH supports all reasonable accommodations, including resources and
services, for students with disabilities, chronic medical conditions, a temporary disability, or a pregnancy complication resulting in difficulties with accessing learning opportunities.  All accommodations are coordinated through the UNC Office of Accessibility Resources \& Services (ARS), \href{https://ars.unc.edu/}{\tt\bf https://ars.unc.edu/}; phone 919-962-8300; email \href{ars@unc.edu}{\tt\bf ars@unc.edu}. Students must document/register their need for accommodations with ARS before accommodations can be implemented.

\subsection*{Counseling and Psychological Services}
CAPS is strongly committed to addressing the mental health needs of a diverse student body through timely access to consultation and connection to clinically appropriate services, whether for short or long-term needs. Go to their website: \href{https://caps.unc.edu}{\tt\bf https://caps.unc.edu} or visit their facilities on the third floor of the Campus Health Services building for a walk-in evaluation to learn more. 

\section*{UNC Honor Code}
As a student at UNC-Chapel Hill, you are bound by the university’s Honor Code, through which UNC maintains standards of academic excellence and community values. It is your responsibility to learn about and abide by the code.  All written assignments or presentations (including team projects) should be completed in a manner that demonstrates academic integrity and excellence. Work should be  completed in your own words, but your ideas should be supported with well-cited evidence and theory.  To ensure effective functioning of the Honor System at UNC, students are expected to: 
\begin{enumerate}
\item Conduct all academic work within the letter and spirit of the Honor Code, which prohibits the giving or receiving of unauthorized aid in all academic processes. 
\item Learn the recognized techniques of proper attribution of sources used in written work; and to identify allowable resource materials or aids to be used during completion of any graded work. 
\item Sign a pledge on all graded academic work certifying that no 
unauthorized assistance has been received or given in the completion of the work. 
\item Report any instance in which reasonable grounds exist to believe that a fellow student has violated the Honor Code.  
\end{enumerate}
Instructors are required to report suspected violations of the Honor Code, including inappropriate collaborative work or problematic use of secondary materials, to the Honor Court. Honor Court sanctions can include receiving a zero for the assignment, failing the course and/or suspension from the university.  If you have any questions about your rights and responsibilities, please consult the Office of Student Conduct at \href{https://studentconduct.unc.edu/}{\tt\bf https://studentconduct.unc.edu/}, or consult these other resources: 
\begin{itemize}
\item Honor system module. 
\item UNC library’s plagiarism tutorial. 
\item UNC Writing Center handout on plagiarism. 
\end{itemize}

\section*{Instructor Expectations}
\subsection*{Email}
The instructor will typically respond to email within 24 hours or less if sent Monday through Friday. The instructor may respond to weekend emails, but it is not required of them. If you receive an out of office reply when emailing, it may take longer to receive a reply. The instructor will provide advance notice, if possible, when they will be out of the office.

\subsection*{Discussion Board}
The instructor will be an active reader and will occasionally post throughout the semester to discussion boards on Slack. The group discussion boards will be moderated by the group members unless an issue is brought to the instructor’s attention by a fellow group member.

\subsection*{Grading}
Assignments, projects and discussion board postings will be graded no more than two weeks after the due date. Assignments that build on the next assignment will be graded within one week of the final due date.
Early submissions will not be graded before the final due date.

\subsection*{Collaboration}
Collaboration will be expected and required on all group projects; assignments consider group projects will be explicitly marked as such.  For weekly homework assignments, collaboration with other students is welcome and encouraged with respect to \textbf{all coding-related} aspects of the work.  However, collaboration will be \textbf{not be allowed} with respect to other aspects of the work (interpretation of results, answers needing written responses, etc.).  Examinations will be done remotely with collaboration with others not being allowed in any form.  Collaboration with any outside textual resources (online, digital, or otherwise) as well as all course texts (textbooks, notes, etc.) will be \textbf{fully allowed} for all assignments and examinations (though written responses must be \textbf{written in your own words}).

\subsection*{Syllabus Changes}
The instructor reserves to right to make changes to the syllabus, including project due dates and test dates. These changes will be announced as early as possible.

\section*{Student Expectations}
\subsection*{Appropriate Use of Course Resources}
The materials used in this class, including, but not limited to, syllabus, exams, quizzes, and assignments are copyright protected works. Any unauthorized copying of the class materials is a violation of federal law and may result in disciplinary actions being taken against the student. Additionally, the sharing of class materials without the specific, express approval of the instructor may be a 
violation of the University's Student Honor Code and an act of academic dishonesty, which could result in further disciplinary action. This includes, among other things, uploading class materials to websites for the purpose of sharing those materials with other current or future students. 

\subsection*{Assignments}
Submit all assignments through GitHub Classroom or assignment links located in the weekly modules, syllabus link, or assignments link (if made available by your instructor).  Emailing assignments is not acceptable unless prior arrangements have been 
made. If you are having issues submitting assignments, try a different web browser first. If switching browsers does not work, email or call the instructor for guidance. 

\subsection*{Late Work}
No late work will be excepted. This is a strict rule with no exceptions since the lowest homework score will be dropped

\subsection*{Technical Support}
The UNC Information Technology Services (ITS) department provides technical support 24-hours per day, seven days per week.  If you need computer help, please contact the ITS Help Desk by phone at +1-919-962-HELP (919-962-4357), or by email at \href{help@unc.edu}{\tt\bf help@unc.edu}, or by visiting their website at \href{http://help.unc.edu}{\tt\bf http://help.unc.edu}, or by UNC Live Chat at \href{http://its.unc.edu/itrc/chat}{\tt\bf http://its.unc.edu/itrc/chat}. 

\section*{Student Evaluation}
\subsection*{Course Assignments and Assessments}
This course will include graded assignments and/or exams, with the student's final grade based on the following scale:
\begin{itemize}
\item Homework (30\%)
\item Article Evaluations (10\%)
\item Midterm (30\%)
\item Group Project (30\%)
\end{itemize}

\subsection*{Grading Scale}
For graduate and professional students, final course grades will be determined using the following UNC Graduate School grading scale.  The relative weight of each course component is shown in the list above. 

\begin{center}
\begin{tabular}{c|c|c}
Letter & Description & Points Range \\
\hline
H & High Pass: Clearly excellent graduate work & $\geq 93$ \\
P & Pass: Entirely satisfactory graduate work & $\geq 80$ \\
L & Low Pass: Inadequate graduate work & $\geq 70$ \\
F & Fail & < 70
\end{tabular}
\end{center}

For undergraduate students, final course grades will be determined using the following UNC Undergraduate School grading system.

\begin{center}
\begin{tabular}{c|m{35em}|c}
Letter & Description & Points Range \\
\hline
A & Mastery of course content at the highest level of attainment that can reasonably be expected of students at a given stage of development. The A grade states clearly that the students have shown such outstanding promise in the aspect of the discipline under study that he/she may be strongly encouraged to continue.  & $\geq 93$ \\
\hline
B & Strong performance demonstrating a high level of attainment for a student at a given stage of development. The B grade states that the student has shown solid promise in the aspect of the discipline under study.  & $\geq 80$ \\
\hline
C & A totally acceptable performance demonstrating an adequate level of attainment for a student at a given stage of development. The C grade states that, while not yet showing unusual promise, the student may continue to study in the discipline with reasonable hope of intellectual development.  & $\geq 70$ \\
\hline
D & A marginal performance in the required exercises demonstrating a minimal passing level of attainment. A student has given no evidence of prospective growth in the discipline; an accumulation of D grades should be taken to mean that the student would be well advised not to continue in the academic field.  & $\geq 60$ \\
\hline
F & For whatever reason, an unacceptable performance. The F grade indicates that the student’s performance in the required exercises has revealed almost no understanding of the course content. A grade of F should warrant an advisor’s questioning whether the student may suitably register for further study in the discipline before remedial 
work is undertaken. & $<60$\\
\end{tabular}
\end{center} 

\subsection*{Assignment Descriptions}
\begin{itemize}
\item Homework (30\%)\\
\begin{center}
\begin{tabular}{c|m{10em}|m{10em}|m{10em}}
Criteria & Fully Met & Partially Met & Not Met \\
\hline
Amount (20 points) & 20 points \newline Completed 100\% of the problems assigned & 15-19 points \newline Completed 80-99\% of the problems assigned & 0-14 points \newline Completed <80\% of the problems assigned\\
\hline
Accuracy (80 points) & 72-80 points & 60-71 points & 0-59 points\\
\end{tabular}
\end{center}
\item Article Evaluations (10\%)
\begin{center}
\begin{tabular}{c|m{10em}|m{10em}|m{10em}}
Criteria & Fully Met & Partially Met & Not Met \\
\hline
Amount (10 points) & 10 points \newline Thoughtful and thorough evaluation of article given & 5-9 points \newline Complete response given, though may show incorrect or limited understanding of article & 0-4 points\newline Incomplete or no response given, shows misunderstanding of article.\\
\end{tabular}
\end{center}
\item Midterm (30\%)\\
Point values will be assigned to each question. Each question will be graded based on whether the question is answered correctly.
\item Group Project (30\%)\\
The students will work on an open-ended challenge problem, set up as a competition.
\begin{center}
\begin{tabular}{m{9em}|m{10em}|m{10em}|m{10em}}
Criteria & Fully Met (10) & Partially Met (6-9) & Not Met (0-5)\\
\hline
Content (10 points) & Analysis code written clearly and efficiently with complete documentation.  Project directory on GitHub is clearly organized containing all project files with reproducibility of the analysis easily facilitated and achieved & Analysis code runs without errors though may be lacking in clarity and efficiency in certain aspects.  Project directory may lack in organization clarity or completeness, harming reproducibility & Analysis code runs with one or more errors or is severely lacking in clarity and efficiency, and documentation is incomplete or poorly done.  Project directory may be poorly organized or incomplete, with analysis not meeting reproducibility standards in any fashion.\\
\hline
Subject Knowledge (10 points) & The project demonstrated knowledge of the course content by integrating major and minor concepts throughout. & The project partially demonstrated knowledge of the course content by integrating major and minor concepts throughout. & The project demonstrated very little knowledge of the course content by major and minor concepts throughout.\\
\end{tabular}
\end{center}
\end{itemize}

\section*{Course Schedule}
The instructor reserves to right to make changes to the syllabus, including project due dates and exam dates. These changes will be announced as early as possible. 

\begin{center}
\begin{tabular}{m{10em}|m{20em}}
Week/Session & Topic and Competency\\
\hline
Session 1 & Intro, assessing model accuracy, bias and variance tradeoff\\
\hline
Session 2 & Regression \& classification: linear regression, logistic 
regression, linear/quadratic discriminant analysis, naïve 
bayes\\
\hline
Session 3 & Nonlinearity: polynomial regression, spline, smooth spline\\
\hline
Session 4 & Resampling: cross-validation, bootstrap\\
\hline
Session 5 & Model selection: penalized regression, regularization\\\hline
Session 6 & Tree-based methods: bagging, boosting, random forest\\
\hline
Session 7 & Kernel methods, support vector machine\\
\hline
Session 8 & Unsupervised learning: dimensionality reduction, clustering\\
\hline
Session 9 & Neural networks, deep learning, and big data\\
\hline
\end{tabular}
\end{center}

More detailed schedule available \href{https://github.com/kmdono02/BIOS_635}{\tt\bf here}

\end{document}